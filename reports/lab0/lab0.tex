\documentclass{article}
\usepackage[top=3cm, bottom=3cm, left = 2cm, right = 2cm]{geometry}
\geometry{a4paper}
\usepackage[T1]{polski}
\usepackage[utf8]{inputenc}
\usepackage{titling}
\usepackage{caption}
\usepackage[parfill]{parskip}
\usepackage{hyperref}
\usepackage{multirow}
\usepackage{graphicx}
\usepackage{pgffor}

\renewcommand\maketitlehooka{\null\mbox{}\vfill}
\renewcommand\maketitlehookd{\vfill\null}

\begin{document}

\begin{titlingpage}
    \title{Algorytmy metaheurystyczne\\[1ex] \large Problem komiwojażera euklidesowego. Drzewo MST. Losowe cykle.}
    \author{Karol Janic}
    \date{27 października 2023}

    \maketitle
\end{titlingpage}

\tableofcontents

\newpage

\section{Cel zadania}
Celem zadania jest sprawdzenie dwóch metod podających kandydatów problemu komiwojażera i porównanie ich. Jedną metodą jest generowanie losowego cyklu zaś drugim rozwiązanie oparte na drzewie MST.

\section{Generowanie kandydata}
\subsection{Losowy cykl}
Tworzymy permutację z wierzchołków grafu reprezentującego instancję problemu. Następnie przy użyciu algorytmu Fisher-Yates mieszamy ją. 

\subsection{Kandydat generowany przy użyciu MST}
W grafie reprezentującym instancję problemu przy użyciu algorytmu Prim'a generujemy minimalne drzewo rozpinające. Potem przechodzimy je algorytmem DFS. Kolejność odwiedzania wierzchołków wyznacza kolejne elementy permutacji reprezentującej kandydata.

\section{Wyniki}
Zaimplementowane metody zostały porównane na przykładach z \url{https://www.math.uwaterloo.ca/tsp/vlsi/index.html}. Wyniki prezentują się następująco:
\begin{table}[h!]
    \centering
    \begin{tabular}{|c|c|c|c|c|c|}
        \hline
        \multirow{3}{*}{Przykład} & Suma & Suma wag & Średnia z minimum & Średnia z minimum  & Minimalna suma  \\
        & wag & kandydata & wag kandydatów dla & wag kandydatów dla & wag z 1000 \\
        & MST & opartego o MST & każdych 10 losowań & każdych 50 losowań  & losowych kandydatów \\
        \hline
        xqf131 & 474 & 718 & 4343.43 & 4216.45 & 3984 \\
        \hline
        xqg237 & 897 & 1445 & 11800.29 & 11498.35 & 10899 \\
        \hline
        pma343 & 1179 & 1883 & 34357.82 & 33356.65 & 32441 \\
        \hline
        pka379 & 1151 & 1855 & 35373.18 & 34531.9 & 33280 \\
        \hline
        bcl380 & 1444 & 2319 & 24697.87 & 24267.7 & 23973 \\
        \hline
        pbl395 & 1124 & 1871 & 19206.86 & 18888.85 & 18399 \\
        \hline
        pbk411 & 1180 & 1935 & 21641.72 & 21276.8 & 20701 \\
        \hline
        pbn423 & 1201 & 1918 & 21942.46 & 21592.2 & 21158 \\
        \hline
        pbm436 & 1269 & 2119 & 22510.76 & 22133.8 & 21358 \\
        \hline
        xql662 & 2240 & 3691 & 51185.66 & 50433.2 & 48914 \\
        \hline
    \end{tabular}
    \caption{Porównanie metod generowania kandydatów dla problemu komiwojażera.}
\end{table}

\section{Wnioski}
\begin{itemize}
    \item suma wag kandydata budowanego na podstawie MST nie przekracza dwukrotnej sumy wag krawędzi MST
    \item kandydat budowany na podstawie MST może zostać ulepszony poprzez usunięcie skrzyżowań
    \item suma wag kandydata generowanego w sposób losowy jest zdecydowanie większa od sumy wag kandydata budowanego na podstawie MST
    \item średnia z minimum wag kandydatów losowych jest mniejsza gdy próbkowane grupy są większe
\end{itemize}

\newpage

\section{Wizualizacje rozwiązań}
\def \myArray{xqf131, xqg237, pma343, pka379, bcl380, pbl395, pbk411, pbn423, pbm436, xql662}

\foreach \text in \myArray {
    \subsection{\text}
    \begin{figure}[h!]
        \centering
        \includegraphics[height=6.5cm]{../../plots/\text-mst.png}
    \end{figure}
    
    \begin{figure}[h!]
        \centering
        \includegraphics[height=6.5cm]{../../plots/\text-tsp-mst.png}
    \end{figure}

    \begin{figure}[h!]
        \centering
        \includegraphics[height=6.5cm]{../../plots/\text-rand.png}
    \end{figure}

    \newpage
}

\end{document}
